\documentclass[solutions]{problem-set}

\usepackage{tikz}
\usetikzlibrary{trees}
\usepackage{pgfplots}

\title{Risk Preferences Problem Set}
\author{Prof. Jeffrey Naecker}
\date{}

\begin{document}

\maketitle

\section{Analyze This}
% code for Machina plots due to Jesse Cohen
Consider the following two risky options:

\begin{itemize}
    \item Option A: 50\% chance of \$20, 50\% chance of \$50
    \item Option B: 25\% chance of \$0, 75\% chance of \$50
\end{itemize}

\begin{enumerate}[label=(\alph*)]
\item Draw an appropriate Machina triangle to analyze these two options, and carefully show where the two options are in the triangle.

\begin{soln}
The lower right corner of the triangle should correspond to the \$0 payoff, the lower left to \$20, and the upper left to \$50.  Option A is at $(0, 0.5)$ and option B is at $(0.25, 0.75)$. 

\begin{tikzpicture}
\begin{axis}[
    axis lines = left,
    xlabel = $p_1$,
    ylabel = {$p_3$},
]


%Budget
\addplot [
    domain=0:1, 
    samples=100, 
    color=black,
]
{-x + 1};

\filldraw[black] (250,750) circle (2pt) node[anchor=west] {B};
\filldraw[black] (0,500) circle (2pt) node[anchor=west] {A};

]
{(1.72)*x};
 
\end{axis}
\end{tikzpicture}

\end{soln}

\item Consider an expected utility decision-maker whose value function is given by $u(x) = x^{\frac{1}{2}}$.  Calculate the utility of the two options, and indicate which one the consumer will choose.

\begin{soln}
We can calculate the utilities as follows:
\begin{align*}
EU_A &= 0.5 \sqrt{20} + 0.5 \sqrt{50} = 5.77 \\
EU_B &= 0.25 \sqrt{0} + 0.75 \sqrt{50} = 5.30
\end{align*}
Thus the consumer prefers option A.
\end{soln}

\item Draw and give the slope for the indifference curves for the consumer in the previous part.  

\begin{soln}
We already know that the indifference curves are straight lines with positive slope.  The slope is given by
\[
\frac{\sqrt{20} - \sqrt{0}}{\sqrt{50} - \sqrt{20}} = 1.72.
\]
\begin{tikzpicture}
\begin{axis}[
    axis lines = left,
    xlabel = $p_1$,
    ylabel = {$p_3$},
]


%Budget
\addplot [
    domain=0:1, 
    samples=100, 
    color=black,
]
{-x + 1};

\filldraw[black] (250,750) circle (2pt) node[anchor=west] {B};
\filldraw[black] (0,500) circle (2pt) node[anchor=west] {A};

%Indifference Curve
\addplot [
    domain=0:.5, 
    samples=100, 
    color=red,
]
{(1.72)*x + .2};

%Indifference Curve
\addplot [
    domain=0:.4, 
    samples=100, 
    color=red,
]
{(1.72)*x+.4};
 
%Indifference Curve
\addplot [
    domain=0:.58, 
    samples=100, 
    color=red,
]
{(1.72)*x};

% A 
\addplot [mark=*, color=black] table {
0 .5 
};

% B
\addplot [mark=*, color=black] table {
.25 .75
};
 
\end{axis}
\end{tikzpicture}
\end{soln}

	
\end{enumerate}


\section{Analyze That}

Consider the following two risky options we saw  earlier:

\begin{itemize}
    \item Option A: 50\% chance of \$20, 50\% chance of \$50
    \item Option B: 25\% chance of \$0, 75\% chance of \$50
\end{itemize}

Consider a prospect theory decision-maker which preference described by $\alpha = 0.5$, $\lambda = 1.5$, $c_r=10$, and $\gamma  = 0.3$.  Calculate their utility of the two options, and indicate which one they will choose.
	
\begin{soln}
First, note that using the probability weighting function we can find that $\pi(0.25) = 0.42$, $\pi(0.5) = 0.5$, and $\pi(0.75) = 0.58$.  We can then calculate the utilities as follows:
\begin{align*}
PT_A &= 0.5 \sqrt{20-10} + 0.5 \sqrt{50-10} = 4.74 \\
PT_B &= 0.42 \left [-2 \sqrt{|0-10|}\right ] + 0.58 \sqrt{50-10} =1.01 % should be 1.5 not 2?
\end{align*}
Thus this consumer prefers option A as well.
\end{soln}	

\end{document}